\rightline{Task 7.4CR Custom Program Design}
\leftline{\bf Design Overview for \it The BBC BASIC MAID \rm}
\leftline{Name: Shade H. St Claire}
\leftline{Student ID: 760657}
\leftline{\bf Summary of Program}

The British Broadcasting Corporations Beginner's All-purpose Symbolic Instruction Code (BBC BASIC) was first designed for the BBC Micro in 1981. BASIC
here doubled both as a basic Operating System as well as a programming language for basic tasks. Similar to how SH or CMD might be percieved within a dumb
terminal. This document, and subsequently the project which it describes is to be a subset of BBC BASIC V developed for Acord risc computers according
to the Archimedes 400 Series BBC BASIC manual which this project treats as its reference. Again, BASIC serves as a `mini OS'. The compiler which is
hereon described is intended to be... well a compiler. Divorced from this abition of kernel abstraction.

However, provided time allows, it is not impossible to imagine a JIT system by which user input may be quickly converted to bytecode and then executed
before prompting again. The major issues here though are that Python cannot simply interface the C-programmed VM as it would be innordinantly unsafe and
slow to use system for this purpose. Another such issue is python's inherent trouble keeping up with compiled languages. I might propose embedding the
VM as a library for this purpose, but such bridges may be crossed when they are reached.

The Limited BASCI Machine Abstraction Interface Directive (or LBMAID) is the instruction set which has been developed for this project. 


\leftline{\bf References}

\leftline{\sl BBC BASIC --- Wikipedia : \rm en.wikipedia.org/wiki/BBC_BASIC}
\leftline{\sl BBC BASIC Versions --- mdfs : \rm mdfs.net/Software/BBCBasic/Versions}
\leftline{\sl Archimedes 400 Series BBC BASIC GUIDE --- Acorn Software : \rm BASIC manual of 1988}
