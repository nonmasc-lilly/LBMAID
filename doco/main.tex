\def\bcode{\begingroup\hrule\hbox\bgroup\vrule\hskip10pt\hsize=440pt\vbox\bgroup\smallskip}
\def\ecode{\smallskip\egroup\vrule\egroup\hrule\endgroup}
\def\entry #1 #2 [#3]{\begingroup\let\par\endgraf\vbox{\hbox to \hsize{\tt\hsize=439pt\vbox{\tt#1 \$#2}\hfil}\hbox{\hskip10pt\vbox{#3}}\hfil}\endgroup}


\hsize=450pt

\centerline{\bf Limited Basic Machine Application Interface Directions}
\centerline{\sl LBMAID}
\centerline{\tt Shade H. St Claire}

\noindent LBMAID is the virtual machine/bytecode to be used in my prospected project implementing a subset of BBC BASIC. It is a limitted form of a much
larger register machine first developed based on the Intel 8086. It thus inherets some of the quirks associated with the intel line of machines while
(hopefully) being compact enough to be easily implemented over the course of a half-week.

LBMAID consists of 9 which various instructions reference both indirectly or directly. Being a 32 bit bytecode designed to run on
the test machine (a Toshiba satellite from 2009 running Windows 10 to... arguable success) the registers too are 32 bit. These registers are as so:

\halign{\item{$\circ$}\tt # \rm & # & #\hfil \cr
        0000 ACC & : & The accumulator.                 \cr
        0001 B   & : & General purpose/array register.  \cr
        0010 C   & : & Counter register.                \cr
        0011 XI  & : & X index register.                \cr
        0100 YI  & : & Y index register.                \cr
        0101 SP  & : & Stack pointer.                   \cr
        0110 BP  & : & Stack base pointer.              \cr
        0111 FLG & : & Flags register.                  \cr
        1000 IP  & : & Instruction pointer.             \cr
}

\noindent The use of these registers will become clear throughout the continuation of this document. We can begin the specifications ahead with that of a
brief description of MAID-ASM, the assembly language assumed for the rest of this document. Its instructions are of the following form:

\medskip
\bcode
        \leftline{\tt   instr = (instruction | directive) *(arguments *SP) comment}
        \leftline{\tt comment = ';' *WHATEVER NL | '/*' *WHATEVER '*/'}
\ecode
\medskip

\noindent An example might be:

\medskip
\bcode
        \leftline{\tt LDI ACC 48h}
        \leftline{\tt PUTCHAR}
        \leftline{\tt LDI ACC 49h}
        \leftline{\tt PUTCHAR}
        \leftline{\tt LABEL .loop}
        \leftline{\tt GETCHAR}
        \leftline{\tt PUTCHAR}
        \leftline{\tt CMP ACC 3h /* 3h is Ctr-C */}
        \leftline{\tt JNZ .loop}
\ecode
\medskip

\noindent Which would print `\tt HI\rm' and echo user input until Ctr-C is pressed. LBMAID does make quite a few concessions for the sake of time, for
example many expected syntaxes are not to be present in this first version of LBMAID, i.e. \tt identifier: \rm for labels is not surported, nor is:
\tt LOD ACC [B + 5] \rm for pointer dereference. Instead we might have: \tt LABEL identifier \rm or \tt LOD ACC +B 5 \rm.

First we will define the various directives:

\halign{\indent\tt # \rm & # & #\hfil\cr
        LABEL \it name\/\tt     & : & Defines a label with the name \it name\/ \rm which is at compile time replaced\cr
                                &   & with the current address in all immediate/memory calculations.\cr
        ORG \it number\/\tt     & : & Sets the current address to \it number\rm.\cr
        \$                      & : & Replaced by the current address at compile time.\cr
        MEM \it amount\/\tt     & : & Sets the amount of allocated memory to be at least \it amount\rm.\cr
        DB [\it bytes\tt]       & : & Defines the bytes at the current address as though an instruction had\cr
                                &   & evalutated to them.\cr
        DW [\it words\tt]       & : & Similar to \tt DB \rm except that values are in little endian double-byte format.\cr
        DD [\it dwords\tt]      & : & Similar to \tt DB \rm except that values are in little endian quat-byte format.\cr
}

\noindent Before the instructions are defined it bears mentioning that the header consists of a quad-byte which determines the initial allocation memory
(minimum) and is defined by the assembler automatically. This means all code is offset in binary by four bytes. It also bears mentioning that code is
initially loaded at address 0x00000000 without regard for stack size/needs. Considering that the stack must grow downwards it is highly advised to move
the code to an appropriate spot at runtime using a block copy command (\tt REP MOVS BYTE\rm).

Now to define the instructions. These instructions are of the form:

\medskip
\leftline{\hskip10pt\tt NAME \$MAX\_ARG\_NUM (ARGUMENT\_TYPES)\hfill}
\medskip

Where MAX\_ARG\_NUM is the maximum number of arguments which may be passed to the instruction where ARGUMENT\_TYPES is a list of argument type signiatures.
Each argument may be one of the following:

\halign{\item{$\circ$}\tt # \rm & # & #\hfill\cr
        R       & : & Any register.\cr
        I       & : & Immediate.\cr
        M       & : & Memory.\cr
        SHORT   & : & 8  bit immediate.\cr
        NEAR    & : & 16 bit immediate.\cr
        FAR     & : & 32 bit immediate.\cr
        BYTE    & : & The keyword \tt BYTE\rm.\cr
        WORD    & : & The keyword \tt WORD\rm.\cr
        DWORD   & : & The keyword \tt DWORD\rm.\cr
        NULL    & : & An empty argument.\cr
        A       & : & Accumulator.\cr
        B       & : & B register.\cr
        C       & : & C register.\cr
        XI      & : & XI register.\cr
        YI      & : & YI register.\cr
        SP      & : & SP register.\cr
        BP      & : & BP register.\cr
        F       & : & FLG register.\cr
        IP      & : & IP register.\cr
}

\noindent Memory access is also of one of the following modes:

\halign{\item{$\circ$}\tt # \rm & # & #\hfill\cr
        0000 immediate          & : & Value at the address of the provided immediate.\cr
        0001 B                  & : & Value pointed to by B.\cr
        0010 XI                 & : & Value pointed to by XI.\cr
        0011 YI                 & : & Value pointed to by YI.\cr
        0100 BP                 & : & Value pointed to by BP.\cr
        0101 + B immediate      & : & Value at B + the immidiate provided.\cr
        0110 + XI immediate     & : & Value at XI + the immediate provided.\cr
        0111 + BP immediate     & : & Value at BP + the immediate provided.\cr
        1000 $\langle$ B        & : & Value at B $\langle\langle$ 1.\cr
        1001 $\langle\langle$ B & : & Value at B $\langle\langle$ 2.\cr
}

\vfill\break
\noindent Finally we may consider the instructions of LBMAID. They are as so:

\newcount\entrynum
\def\bentries{\halign\bgroup\tt## & \tt## & \tt##\hfill\cr}
\def\eentries{\egroup}
\def\entry #1 #2 (#3){#1 & #2 & (#3)\cr}

\bentries
\entry 00 ADD (A, I)
\entry 01 ADV (C, I)
\entry 02 ADD (A, M)
\entry 03 ADD (A, B)

\entry 04 AND (A, I)
\entry 05 AND (A, M)
\entry 06 AND (A, B)

\entry 07 CALL (SHORT)
\entry 08 CALL (NEAR)
\entry 09 CALL (FAR)

\entry 0A CMP (A, B)
\entry 0B CMP (A, I)
\entry 0C CMP (A, M)

\entry 0E CMPS (BYTE)
\entry 0F CMPS (WORD)
\entry 10 CMPS (DWORD)

\entry 11 CURPOS ()

\entry 12 FADD (R, R)

\entry 13 FSUB (R, R)

\entry 14 GETCHAR ()

\entry 15 HLT ()

\entry 16 JC (SHORT)

\entry 17 JNC (SHORT)

\entry 18 JZ (SHORT)

\entry 19 JNZ (SHORT)

\entry 1A JO (SHORT)

\entry 1B JS (SHORT)

\entry 1C JMP (SHORT)
\entry 1D JMP (NEAR)
\entry 1E JMP (FAR)

\entry 1F LODS (BYTE)
\entry 20 LODS (WORD)
\entry 21 LODS (DWORD)

\entry 22 LOOP (SHORT)

\entry 23 LOOPZ (SHORT)

\entry 24 LOOPNZ (SHORT)

\entry 25 LOD (A, M)
\entry 26 LOD (C, M)
\entry 27 LOD (R, M)

\entry 28 LDI (A, I)
\entry 29 LDI (R, I)

\entry 2A MOV (A, R)
\entry 2B MOV (R, R)
\entry 2C MOV (SP, BP)
\entry 2D MOV (BP, SP)

\entry 2E MOVS (BYTE)
\entry 2F MOVS (WORD)
\entry 20 MOVS (DWORD)

\entry 21 NOT (A)

\entry 32 OR (A, I)
\entry 33 OR (A, R)

\entry 34 POP (A)
\entry 35 POP (M)
\entry 36 POP (BP)

\entry 37 POPF ()

\entry 38 PUSH (A)
\entry 39 PUSH (I)
\entry 3A PUSH (M)
\entry 3B PUSH (BP)

\entry 4C REP/REPZ ()

\entry 3D REPNZ ()

\entry 3E RET (SHORT)

\entry 3F SCAS (BYTE)
\entry 40 SCAS (WORD)
\entry 41 SCAS (DWORD)

\entry 42 STO (M, A)
\entry 43 STO (M, I)

\entry 44 STOS (BYTE)
\entry 45 STOS (WORD)
\entry 46 STOS (DWORD)

\entry 47 SUB (A, I)
\entry 48 SUV (C, I)
\entry 49 SUB (A, M)
\entry 4A SUB (A, B)

\entry 4B TEST (A, B)
\entry 4C TEST (A, I)
\entry 4D TEST (A, M)

\entry 4E ZERO (R)

\entry 4F XOR (A, B)

\entry 50 PUTCHAR ()

\entry 51 PUSHF ()
\cr
\eentries

\noindent An example program follows:

\medskip
\bcode
        \leftline{\tt LDI XI message}
        \leftline{\tt /* Size optimized instructions to load string length into C and increment XI to}
        \leftline{\tt\qquad the appropriate length */}
        \leftline{\tt LOD C \% XI ; \% preceeds mem arguments for clarity}
        \leftline{\tt LODS DWORD}
        \leftline{\tt LABEL .loop}
        \leftline{\tt\qquad LODS BYTE}
        \leftline{\tt\qquad PUTCHAR}
        \leftline{\tt LOOP .loop}
        \leftline{\tt HLT}
        \leftline{\tt LABEL message}
        \leftline{\tt DD 0Ch\% ; \% also proceeds lists}
        \leftline{\tt DB 'Hello World!'\%}
\ecode

\bye
